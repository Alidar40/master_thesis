\Introduction
% Введение содержит четкое и краткое обоснование выбора темы, ее актуальности, определение
% объекта и предмета исследования, целей и задач, перечень методов исследования; краткую
% формулировку научно-теоретической и практической значимости исследования; сведения об
% апробации результатов исследования (публикации, выступления на конференциях и т.д.).

% Объектом исследования является перенос стиля текста.
% Предметом исследования являются архитектурные особенности нейронных сетей.

% Теоретическая значимость работы заключается в 

% Практическая -- результаты исследования могут быть использованы для повышения качества переноса.

% С ростом популярности больших генеративных языковых моделей, управление его генерацией является
% В последние годы исследования стиля текста привлекают внимание не только лингвистов, но и многих исследователей в области компьютерных наук.
% В частности, исследователи компьютерных наук исследуют задачу переноса стиля текста, которая становится все более популярной ветвью генерации естественного языка, целью которой является изменение стилистических свойств текста при сохранении его независимого содержания стиля.

% В последние годы в компьютерной лингвистике исследуют задачу переноса стиля текста, которая становится все более популярной ветвью генерации естественного языка, целью которой является изменение стилистических свойств текста при сохранении его независимого содержания стиля.

Возможность управлять стилем текста является очень актуальной и важной задачей, потому что это делает обработку естественного языка более ориентированной на пользователя.
Алгоритмы переноса стиля текста имеют множество непосредственных применений.
Например, одним из таких приложений являются интеллектуальные боты, для которых
пользователи предпочитают эмоциональную и последовательную персону.
Другим применением является разработка интеллектуальных помощников по написанию текстов;
например, авторам-неспециалистам часто требуется улучшить свои тексты, чтобы они лучше соответствовали их назначению, например, более профессиональному, вежливому, объективному, с чувством юмора или другим продвинутым требованиям к письму, на освоение которых могут потребоваться годы опыта.
Другие приложения включают автоматическое упрощение текста, увеличение объективности онлайн-текста, борьбу с оскорбительными выражениями и токсичными высказываниями и т.д.

% Данная задача, хоть и напоминает другие проблемы компьютерной лингвистики, например, машинный перевод или парафраз, является более сложной в связи с отсутствием достаточного объёма параллельных данных.
Данная задача близка к другим задачам компьютерной лингвистики, например, машинному переводу или парафразу.
Поэтому методы и экспертиза, накопленные в решении этих проблем, могут применяться и для неё.

Для выполнения поставленной цели, необходимо решить следующие задачи:
\begin{itemize}
    \item проанализировать текущее состояние исследуемой области, существующие подходы к решению данной или схожей задачи, имеющиеся наборы данных;
    \item определить необходимые метрики и критерии оценки качества будущего алгоритма;
    \item собрать собственный набор данных;
    \item провести эксперименты для проверки работоспособности выбранных методов на собранном наборе данных;
    \item сделать выводы на основании полученных результатов;
    \item сформулировать планы для дальнейшего развития исследования.
\end{itemize}
