\chapter*{Аннотация}

% Объем аннотации должен составлять не более 1500 знаков. В аннотации необходимо
% отразить:
% - цели и задачи работы;
% - полученные результаты;
% - рекомендации, предложенные на основании данной работы.

Целью данной работы является разработка алгоритма переноса стиля из неформального стиля, присущего интернет общению, в стиль формальный и наоборот.

В ходе выполнения исследования был собран параллельный набор данных из 18 тысяч параллельных пар предложений неформальный-формальный стиль.
В основе набора данных лежит корпус текстов статей с сайта Луркморье.
Для разметки данных были привлечены студенты-лингвисты из Российского Государственного Гуманитарного Университета.

В работе были исследованы 4 различных метода, как основанные на полном отсутствии параллельных данных, так и с привлечением современных языковых моделей.
Подходы, основанные на использовании больших языковых моделей, оказались наиболее эффективными для решения поставленной задачи.
Данные методы, совместно с параметро-эффективными методами обучения, не требуют наличия большого количества параллельных данных, а с увеличением количества параметром языковой модели, объём необходимого количества данных для дообучения сводится до минимума.
