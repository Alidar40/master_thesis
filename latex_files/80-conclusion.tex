\Conclusion % заключение к отчёту

% Заключение – последовательное логически стройное изложение итогов исследования в
% соответствии с целью и задачами, поставленными и сформулированными во введении. Заключение
% может включать в себя практические предложения, что повышает ценность теоретического
% материала, но не должно повторять введение
Был собран набор данных для переноса формальности на русском языке объёмом 18 тысяч параллельных пар формальное-неформальное предложение.
Данного набора данных достаточно, для оценки качества моделей, а так же их обучению в случае использования параметро-эффективных методов обучения с современными языковыми моделями.

Можно сделать вывод, что лучше всего в решении поставленной задачи показали себя большие языковые модели, способные без какого-либо дообучения на конкретную задачу показывать результат лучше, чем целенаправленно обученные на задачу модели.
Языковые модели прошлого поколения (GPT-2, GPT-3, T-5 и пр.) с помощью параметро-эффективных методов обучения могут показать хороший результат, особенно в случае отсутствия большого набора данных.
Более подздние алгоритмы показывают качество генераций намного хуже.

Обучение без учителя и Guided Generation показали лучшие метрики SacreBLEU и METEOR.
Использование предобученных языковых моделей в сочетании с параметро-эффективными методами дообучения и использование собранного параллельного набора данных дало наилучший результат с точки зрения комбинации метрик Style Score и Semantic Score.
Это подтверждает другие исследования, что использование метрик, основанных на n-граммах и сравнении с референсами, зачастую не даёт адекватной оценки качества алгоритма.
Данное наблюдение поднимает вопрос о необходимости переосмысления подходов к оценке качества алгоритмов переноса стиля.

% TODO ВЫВОДЫ ИЗ ВВЕДЕНИЯ СТРУКТУРИРОВАНО
% Для выполнения поставленной цели, необходимо решить следующие задачи:
% \begin{itemize}
%     \item проанализировать текущее состояние исследуемой области, существующие подходы к решению данной или схожей задачи, имеющиеся наборы данных;
%     были пр
%     \item определить необходимые метрики и критерии оценки качества будущего алгоритма;
%     \item собрать собственный набор данных;
%     \item провести эксперименты для проверки работоспособности выбранных методов на собранном наборе данных;
%     \item сделать выводы на основании полученных результатов;
%     \item сформулировать планы для дальнейшего развития исследования.
% \end{itemize}

