\label{cha:future}
\chapter{Развитие работы}

\section{Продолжение создания набора данных}
% С получением опыта и понимания специфики создания подобного набора данных, имеет смысл продолжить его сбор.
Приняв во внимания аспекты из главы \ref{cha:dataset}, а так же усовершенствование алгоритмов классификации, имеет смысл продолжить создание набора данных на новом качественном уровне.
Усовершенствованные алгоритмы позволят более качественно фильтровать изначальный корпус и выдавать асессорам более качественные предложения.
Тем временем, улучшая взаимодействие с асессорами, можно добиться более высокого итогового качества.

\section{Улучшение метрик качества}
Как было упомянуто в главе \ref{cha:experiments}, данная работа подтвердила другие недавние исследования, указывавшие на проблему соответствия автоматических метрик человеческим оценкам в общем, и в использовании референсных метрик в частности.
Существующие ограничения метрик оценки качества мотивируют к изучению новых способов оценки качества работы моделей по переносу стиля текста.

\section{Улучшение алгоритмов}
Исследование показало, что сбор набора данных хоть и посильная, но очень труднозатратная активность.
В связи с этим дальнейшее изучение и улучшение алгоритмов, не требующих параллельного набора данных, является перспективным направлением.
Особенно в рамках использования больших языковых моделей, показывающих впечатляющий результат, без дообучения на конкретную задачу.
Несмотря на это, большие языковые модели являются чрезмерно громоздкими и работа по уменьшению их размера для использования в задаче переноса стиля текста является обещающим направлением для исследования.
